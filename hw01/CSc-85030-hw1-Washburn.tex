%This is my super simple Real Analysis Homework template

\documentclass{article}
\usepackage{geometry}
\geometry{
	a4paper,
%	total={170mm,257mm},
	left=20mm,
	right=20mm,
	top=20mm,
}
\usepackage[utf8]{inputenc}
\usepackage[english]{babel}
\usepackage{mathrsfs} % https://www.ctan.org/pkg/mathrsfs
\usepackage{amsmath}
\usepackage[]{amsthm} %lets us use \begin{proof}
\usepackage[]{amssymb} %gives us the character \varnothing
\usepackage{listings}
\usepackage{enumitem}
\usepackage{censor}
\usepackage{comment}
\usepackage{physics}
\usepackage{algorithm}
\usepackage[noend]{algpseudocode}

\lstnewenvironment{scala}{\lstset{language=Scala}}{}

%%% REDACT the problem statements that the grader doesn't like!
%\newcommand{\HomeworkText}[1]{\xblackout{``#1``}\\}
%\excludecomment{HomeworkScala}
\newcommand{\HomeworkText}[1]{\textbf{``#1''}\\}
\lstnewenvironment{HomeworkScala}{\lstset{language=Scala}}{}

\setlength{\abovedisplayskip}{1pt}
\setlength{\belowdisplayskip}{1pt}
\setlength{\abovedisplayshortskip}{1pt}
\setlength{\belowdisplayshortskip}{1pt}

\newcommand*{\defeq}{\stackrel{\text{def}}{=}}

\title{CSc - 850 Modern Cryptography\\ Homework 1}
\author{Alex Washburn}
\date\today

\begin{document}
	\newgeometry{left=4.5cm,right=4.5cm}

	\maketitle
	\vspace*{4cm}

	\restoregeometry
	
	\clearpage
	\section*{Question 1}
	\HomeworkText{
		Prove or disprove the following:
	}

	\subsection*{Part A}
	\HomeworkText{
		There is a perfectly secret encryption scheme that reveals 99\% of the key.
	}

    Let $\Pi = (Gen,Enc,Dec)$ be a perfectly secure encryption scheme with $2^\ell = |\mathcal{K}| = |\mathcal{M}| = |\mathcal{C}|$.
    Note that by \textbf{Theorem 2.11(1)} (Shannon's Theorem) on page 36 of the assigned textbook, \emph{``Introduction to Modern Cryptography, 2nd Edition''}, every key $k \in \mathcal{K}$ is chosen with probability $2^{-\ell}$ by algorithm $Gen$.
    Since we will use $\Pi$ in the constructive proof which follows, we should note that for all $\ell \in \mathbb{Z}^{+}$ there exists at least one such encryption scheme satisfying $\Pi$, specifically the One Time Pad for messages of length $\ell$.
    Now we will prove the existence of the hypothesized encryption scheme.\\
    
    Let $\Pi^{*} = (Gen^{*},Enc^{*},Dec^{*})$ be an encryption scheme with $2^{99\cdot\ell}\cdot2^\ell = 2^{100\cdot\ell} = |\mathcal{K}^{*}|$ and $\mathcal{M}^{*} = \mathcal{M}$ and $\mathcal{C}^{*} = \mathcal{C}$.
        
    Let $f : \mathcal{K}^{*} \to \mathcal{K}$ with the following properties:
    \begin{enumerate}

    	\item $f$ is a surjective function where each element in $\mathcal{K}$ has \emph{exactly} $2^{100}$ elements in $\mathcal{K}^{*}$ mapped to it by $f$.

    	\item For each element $k \in \mathcal{K}$, and for all $k^{*} \in \mathcal{K}^{*}$ such that $f(k^{*}) = k$, the correlation $corr(k,k^*)$ is $0$.
    	 
    \end{enumerate}
    
    Let $Gen^{*}$ sample all $k^{*} \in \mathcal{K}^{*}$ with probability $2^{-100\cdot\ell}$.
    
    Let $Enc^{*}_{k^*}(m) = Enc_{f(k^*)}(m)$
    
    Let $Dec^{*}_{k^*}(c) = Dec_{f(k^*)}(c)$
   
    Let $\Psi : \mathcal{C} \to \mathcal{K}^{*}$ with the following properties:
    
    \begin{enumerate}
    
    	\item $\forall\, k^{*} \in \mathcal{K}^{*}, \forall\, m \in \mathcal{M} \quad \Psi(Enc^{*}_{k^*}(m))$ and $k^*$ have $99\cdot\ell$ bits in common.
    	
    	\item $\forall\, k^{*}_0, k^{*}_1 \in \mathcal{K}^{*}, \forall\, m \in \mathcal{M} \;\;$ the correlation $corr(\Psi(Enc^{*}_{k^*_0}(m)),\; \Psi(Enc^{*}_{k^*_1}(m)))$ is $0$.
    	
    	\item $\forall\, k^{*} \in \mathcal{K}^{*}, \forall\, c \in \mathcal{C} \;\;$ the correlation $corr(\Psi(c),k^{*})$ is $0$.
   
    \end{enumerate}

    Note that $Gen^* \xrightarrow{f} \mathcal{K}$ samples all $k \in \mathcal{K}$ with probability $\cdot2^{-\ell}$. Consequently,\\ $$\forall (k^*,m,c) \in \mathcal{K^*} \cross \mathcal{M} \cross \mathcal{C}\;\; \text{Pr}[Enc^*_{k*}(m) = c] = 2^{-\ell} = \text{Pr}[Dec^*_{k*}(c) = m]$$. This follows directly from the definitions of $f$, $Enc^*$, and $Dec^{*}$.\\

	Let $\mathcal{K}'(k^*) \defeq \left\{ x  \;|\; x \in \mathcal{K^*} \text{ where } x \text{ and } k^* \text{ have } 99\cdot\ell \text { bits in common} \right\}$
    Note that $\forall k^* \in \mathcal{K}^*,\; |\mathcal{K}'(k^*)| = 2^\ell$. %Hence $\forall k' \in \mathcal{K}^*(k^*), k' \not = k^*, \text{Pr}[Enc^{*}_{k^*}(m) ]$
    
  	Let $\mathcal{M}'(c) \defeq \left\{ x  \;|\; x = Dec^{*}_{k^*}(c) \text { for some } k^* \in \mathcal{K'}(\Psi(c))\right\}$
  	
  	Let $\mathcal{M}(c) \defeq \left\{ x  \;|\; x = Dec_k(c) \text { for some } k \in \mathcal{K} \right\}$
  	
  	Note that $\forall c \in \mathcal{C},\; |{M}'(c)| = |{M}(c)|$\\
    

    
    We now have all the requisite components for our proof:\\
    
    $\forall m,m' \in \mathcal{M},$ and $\forall c \in \mathcal{C}$.
    
   	\begin{equation}
    \begin{split}
    	\text{Pr}[Enc^*_{k*}(m) = c \;|\; k' = \Psi(c)]
    & = \frac{|{M}'(c)|}{|\mathcal{M}|} \\
    & = \frac{|{M}(c)|}{|\mathcal{M}|} \\
    & = 2^{-\ell} \\
    & = \text{Pr}[Enc^*_{k*}(m') = c]
    \end{split}
    \end{equation}
    
    Therefore $\Pi^*$ is a perfectly secret encryption scheme by \textbf{Lemma 2.4} on page 30.
	

	\subsection*{Part B}
	\HomeworkText{
		There is an encryption scheme that is not perfectly secure, yet no adversary can guess the key with probability better than $2^{\lambda}$, where $\lambda$ is the size of the key.
	}

    Let $\Pi = (Gen,Enc,Dec)$ be a perfectly secure encryption scheme with $\lambda = |\mathcal{K}| = |\mathcal{M}| = |\mathcal{C}|$.
    Note that by \textbf{Theorem 2.11(1)} on page 36 (Shannon's Theorem), every key $k \in \mathcal{K}$ is chosen with probability $2^{-\lambda}$ by algorithm $Gen$.
    Since we will use $\Pi$ in the constructive proof which follows, we should note that for all $\lambda \in \mathbb{Z}^{+}$ there exists at least one such encryption scheme satisfying $\Pi$, specifically the One Time Pad for messages of length $\lambda$.
    Now we will prove the existence of the hypothesized encryption scheme.\\ 
    
    Choose some $n \in \mathbb{Z}^{+}$.
    
    We will utilize  $n$ and $\Pi$ to construct $\Pi^{*} = (Gen,Enc^{*},Dec^{*})$ with $\mathcal{K}^{*} = \mathcal{K}$ and $\lambda + n = |\mathcal{M}^{*}| = |\mathcal{C}^{*}|$. 
    
    $Gen$ is the same in both $\Pi$ and $\Pi^{*}$. 
    
    $Enc^{*}(m) : \mathcal{M} \to \mathcal{C} = Enc(m_{[0,\lambda-1]}) \,||\, m_{[\lambda,n-1]}$; applying $Enc$ on the first $\lambda$ bits of $m$ and leaving the last $n$ bits of $m$ unchanged. 
    
    $Dec^{*}(c) : \mathcal{C} \to \mathcal{M} = Dec(c_{[0,\lambda-1]}) \,||\, c_{[\lambda,n-1]}$; applying $Dec$ on the first $\lambda$ bits of $c$ and leaving the last $n$ bits of $c$ unchanged.\\
    
    We can see, trivially that $\Pi^{*}$ is not a perfectly secret encryption scheme as it leaks the trailing $n$ bits of the message, violating the requirement of perfect indistinguishablity. However, $\forall\, k \in \mathcal{K}^{*}$ and all adversaries $\mathcal{A}$, the adversary's guessed key $K$ is equal to the actual key $k$ with probability $2^{-\lambda}$ because $\mathcal{K}^{*} = \mathcal{K}$.    
    Therefore we have a constructive proof that for all $\lambda \in \mathbb{Z}^{+}$ there exists a countably infinite number of encryption schemes which are not perfectly secure, yet no adversary can guess the key with probability better than $2^{-\lambda}$, where $\lambda$ is the size of the key.


	\clearpage
	\section*{Question 2}
	\HomeworkText{
		State whether they each function is negligible or not and give a brief explanation.
	}

	\subsection*{Part A
	\HomeworkText{
		$f(n) = n^{-100}$
	}}

	$f$ is \emph{not} negligible. Consider the polynomial function $g(n) = n^{102}$
	
	\begin{equation}
	\begin{split}
	\pdv{}{n} \frac{f(n)}{g^{-1}(n)} & = \pdv{}{n} \frac{n^{-100}}{n^{-102}} \\
	  & = \pdv{}{n} n^2 \\
	  & = 2n \\
	  & \ge 1
	\end{split}
	\end{equation}
	
	We can see that the inverse of $g$ decreases faster than $f$ because the derivative of their ratio is greater than or equal to $1$.

	\subsection*{Part B
	\HomeworkText{
		$f(n) = 2^{-100 \log n}$
	}}

	$f$ is \emph{not} negligible. Consider the following equality:
	
	$$ f(n) = 2^{-100 \log n} = n^{-100}$$
	
	We can see above from \textbf{Part A} that $f$ is not negligible.
	
	\subsection*{Part C
	\HomeworkText{
		$f(n) = n^{-\log n}$
	}}

	$f$ is negligible. The function $f$ is presented in the assigned textbook, \emph{``Introduction to Modern Cryptography, 2nd Edition''}, on page 49 as an example of a negligible function. However, instead of taking the book's assertion at face value without a proof, we will sketch out our own.\\
	
	Choose a constant $c \in \mathbb{R}^{+}$. Let $N = 2^{c+1}$. Then for all $n \ge N$ the inequality $n^{-\log n} < n^{-c}$ will hold.\\
	
	\begin{equation}
	\begin{split}
	  	n^{-\log n} & < n^{-c} \\
	  	(2^{c+1})^{-\log(2^{c+1})} & < (2^{c+1})^{-c} \\
	    (2^{c+1})^{-c - 1} & < (2^{c+1})^{-c} \\
	\end{split}
	\end{equation}


	\subsection*{Part D
	\HomeworkText{
		$f(n) = poly(n) \cdot negl(n)$
	}}

	$f$ is negligible. The function $f$ is defined as neglible by \textbf{Proposition 3.6(2)} presented in the assigned textbook, \emph{``Introduction to Modern Cryptography, 2nd Edition''}, on page 49. Let's sketch a a proof again.\\
	
	Assume $f(n) = g(n) \cdot negl(n)$, where $g(n)$ is some positive polynomial function, is not negligible. Note that by this definition neither $g(n)$ nor $g^{-1}(n)$ can be negligible. We quickly arrive at the following contradiction:
	
	\begin{equation}
	\begin{split}
	f(n) & = g(n) \cdot negl(n) \\
	g^{-1} \cdot f(n) & = g^{-1} \cdot g(n) \cdot negl(n) \\
	g^{-1} \cdot f(n) & = negl(n) \\
	\end{split}
	\end{equation}
	
	\clearpage
	\section*{Question 3}
	\HomeworkText{
		Show that $A$ returns the correct bit $b$ with probability $\frac{1}{2} + \frac{\epsilon}{2}$ if and only if $A$ can distinguish the distributions $D_0$, $D_1$ with probability $\epsilon$.
	}

    $$\underset{ x \gets D_0}{\Pr}[A(x) = 1] \;- \underset{ x \gets D_1}{\Pr}[A(x) = 1] = \epsilon \iff \underset{\substack{ b \gets \{0,1\}\\x \gets D_b}}{\Pr}[ A(x) = b] = \frac{1}{2} + \frac{\epsilon}{2}$$
    
    \begin{equation}
    \begin{split}
    \underset{\substack{ b \gets \{0,1\}\\x \gets D_b}}{\Pr}[ A(x) = b] = \frac{1}{2} + \frac{\epsilon}{2} & \equiv \underset{\substack{ b \gets \{0,1\}\\x \gets D_b}}{\Pr}[ A(x) = b] - \frac{1}{2} = \frac{\epsilon}{2}\\
    & \equiv \frac{1}{2}\underset{ x \gets D_0}{\Pr}[A(x) = 0 \;|\; b = 0] \; + \frac{1}{2}\underset{ x \gets D_1}{\Pr}[A(x) = 1 \;|\; b = 1] - \frac{1}{2} = \frac{\epsilon}{2} \\
    & \equiv \underset{ x \gets D_0}{\Pr}[A(x) = 0 \;|\; b = 0] \; + \underset{ x \gets D_1}{\Pr}[A(x) = 1 \;|\; b = 1] - 1 = \epsilon \\
    & \equiv \frac{\underset{ x \gets D_0}{\Pr}[b = 0 \;|\; A(x) = 0] \underset{ x \gets D_0}{\Pr}[A(x) = 0]}{\Pr[b = 0]} + \frac{\underset{ x \gets D_1}{\Pr}[b = 1 \;|\; A(x) = 1] \underset{ x \gets D_1}{\Pr}[A(x) = 1]}{\Pr[b = 1]} - \frac{1}{2} = \frac{\epsilon}{2} \\
    & \equiv \frac{2^{-1}(1 - \underset{ x \gets D_0}{\Pr}[A(x) = 1])}{2^{-1}} + \frac{2^{-1}\underset{ x \gets D_1}{\Pr}[A(x) = 1]}{2^{-1}} - 1 = \epsilon \\
    & \equiv 1 - \underset{ x \gets D_0}{\Pr}[A(x) = 1] + \underset{ x \gets D_1}{\Pr}[A(x) = 1] - 1 = \epsilon \\
    & \equiv \underset{ x \gets D_1}{\Pr}[A(x) = 1] - \underset{ x \gets D_0}{\Pr}[A(x) = 1] = \epsilon \\
    \end{split}
    \end{equation}
    
    Well, I arrived at \emph{almost} the desired the desired goal. However, the subtrahend and minuend are swapped. I'd argue that there should be absolute value bars around the difference of probabilities when distinguishing between the distributions. If that argument is sound, the result of the equivalencies above with the swapped subtrahend and minuend is identical to the desired difference of probabilities.
	

	\clearpage
	\section*{Question 4}
	\HomeworkText{
		Let $p$ be an odd prime.
	}
	\subsection*{Part A}
	\HomeworkText{
		Find an efficient algorithm to check if a given $a \in \mathbb{Z}^{*}_{p}$ is a quadratic residue.
	}

	Given an odd prime $p$ and $p > a \in \mathbb{N}$ decide efficiently if $\exists x \in  \mathbb{N}$ such that $x < p$ and $x^2 \equiv a \mod p$.\\
	
	Select a generator $g$ of $\mathbb{Z}_p$.
	Due to the fact that $g$ generates $\mathbb{Z}_p$, we know that $\exists n \in \mathbb{Z}^{+}$ such that $g^n = a$. \\
	If $n = 2m,\, m \in \mathbb{Z}^{+}$ ($n$ is even), then $a$ is a quadratic residue of $p$ due to the following equality:
	
	$$a = g^{n} = g^{2m} = (g^m)^2 = x^2$$
	
	Given that $g$ generates $\mathbb{Z}_p$ we also know that $g^{p-1} \equiv 1 \mod p$. Observe the following important equivalence:
	
	\begin{equation}
	\begin{split}
	g^{p-1} & \equiv 1 \mod p\\
	(g^{p-1})^m & \equiv 1 \mod p\\
	g^{m(p-1)} & \equiv 1 \mod p\\
	(g^{2m})^{(p-1)\cdot2^{-1}} & \equiv 1 \mod p\\
	(g^n)^{(p-1)\cdot2^{-1}} & \equiv 1 \mod p\\
	a^{(p-1)\cdot2^{-1}} & \equiv 1 \mod p\\
	\end{split}
	\end{equation}
	
	Hence we can decide if $a$ is a quadratic residue of $p$ by first letting $y = \frac{p-1}{2} \in \mathbb{Z}^{+}$. Then compute $a^y \mod p$. If $a^y \equiv 1 \mod p$, output \texttt{True}, otherwise output \texttt{False}. Using binary exponentiation to calculate $a^y \mod p$, the runtime is $\mathcal{O}(y\log a)$. Hence we have an efficient algorithm deciding if $a \in \mathbb{Z}_p^*$ is a quadratic residue.


	\subsection*{Part B}
	\HomeworkText{
		Describe an efficient algorithm to check if a given $g \in \mathbb{Z}^{*}_{p}$ is a generator. Assume that the algorithm is also given a factorization of $p - 1$.
	}

	Let $g \in \mathbb{Z}^{*}_{p}$ be the candidate generator of $\mathbb{Z}^{*}_{p}$.

	Let $\mathcal{F} = \left\{f_0, f_1, ... , f_{n-1}\right\}$ be the $n$ factors of $p-1$.
	
	Let $h(f)$ be the product of $\mathcal{F} \setminus \left\{ f \right\}$.\\
	
	Note that if $g^{h(f)} = 1 \mod p$ then $g$ has order $h(f) < p - 1$.\\

    Now we can assert that $g$ is a generator of $\mathbb{Z}^{*}_{p} \iff \forall\, f \in \mathcal{F},\; g^{h(f)\cdot f} \equiv 1 \mod p\;$ and $\;g^{h(f)} \not \equiv 1 \mod p$. 
    
        \begin{algorithm}
    	\label{Alg:decide-gen}
    	\caption{Decide if $g$ is a generator of $\mathbb{Z}^{*}_{p}$ given the factorization $\mathcal{F}$ of $p - 1$.}
    	\label{Alg:deriveAlignment}
    	\begin{algorithmic}[1]
    		\Function{DecideGenerator}{$g,\, \mathcal{F}$}
    		\State $q \gets \prod_{\mathcal{F}}$
    		\State $p \gets q + 1$
    		\State $h(f) \gets q / f$
    		\ForAll{$f \in \mathcal{F}$}
    		\If{$g^{h(f)} \equiv 1 \mod p$}
    		  \State \Return \texttt{False}
    		\EndIf
    		\EndFor
    		
    		\State \Return \texttt{True}

    		\EndFunction
    	\end{algorithmic}
    \end{algorithm}

	Algorithm \ref{Alg:decide-gen} decides if $g$ is a generator of $\mathbb{Z}^{*}_{p}$ given the factorization $\mathcal{F}$ of $p - 1$ in $\mathcal{O}(p*|\mathcal{F}|)$ time.

\end{document}