%This is my super simple Real Analysis Homework template

\documentclass{article}
\usepackage{geometry}
\geometry{
	a4paper,
%	total={170mm,257mm},
	left=20mm,
	right=20mm,
	top=20mm,
}
\usepackage[utf8]{inputenc}
\usepackage[english]{babel}
\usepackage{mathrsfs} % https://www.ctan.org/pkg/mathrsfs
\usepackage{amsmath}
\usepackage[]{amsthm} %lets us use \begin{proof}
\usepackage[]{amssymb} %gives us the character \varnothing
\usepackage{listings}
\usepackage{enumitem}
\usepackage{censor}
\usepackage{comment}
\usepackage{physics}

\lstnewenvironment{scala}{\lstset{language=Scala}}{}

%%% REDACT the problem statements that the grader doesn't like!
%\newcommand{\HomeworkText}[1]{\xblackout{``#1``}\\}
%\excludecomment{HomeworkScala}
\newcommand{\HomeworkText}[1]{\textbf{``#1''}\\}
\lstnewenvironment{HomeworkScala}{\lstset{language=Scala}}{}

\setlength{\abovedisplayskip}{1pt}
\setlength{\belowdisplayskip}{1pt}
\setlength{\abovedisplayshortskip}{1pt}
\setlength{\belowdisplayshortskip}{1pt}

\title{CSc - 850 Modern Cryptography\\ Homework 1}
\author{Alex Washburn}
\date\today

\begin{document}
	\newgeometry{left=4.5cm,right=4.5cm}

	\maketitle
	\vspace*{4cm}

	\restoregeometry
	
	\clearpage
	\section*{Question 1}
	\HomeworkText{
		Prove or disprove the following:
	}

	\subsection*{Part A}
	\HomeworkText{
		There is a perfectly secret encryption scheme that reveals 99\% of the key.
	}

	From \textbf{Theorem 2.10} on page 35 of the assigned textbook, \emph{``Introduction to Modern Cryptography, 2nd Edition''}, we know that any perfectly secret encryption scheme must have a key space that is at least as large as the message space. From \textbf{Definition 2.3} on page 29, we know that $\text{Pr}[M = m | C = c] = \text{Pr}[M = m]$.
	
	Let us fix $l$
	

	\subsection*{Part B}
	\HomeworkText{
		There is an encryption scheme that is not perfectly secure, yet no adversary can guess the key with probability better than $2^{\lambda}$, where $\lambda$ is the size of the key.
	}

    Let $\Pi = (Gen,Enc,Dec)$ be a perfectly secure encryption scheme with $\lambda = |\mathcal{K}| = |\mathcal{M}| = |\mathcal{C}|$.
    Note that by \textbf{Theorem 2.11(1)} on page 36 (Shannon's Theorem), every key $k \in \mathcal{K}$ is chosen with probability $2^{-\lambda}$ by algorithm $Gen$.
    Since we will use $\Pi$ in the constructive proof which follows, we should note that for all $\lambda \in \mathbb{Z}^{+}$ there exists at least one such encryption scheme satisfying $\Pi$, specifically the One Time Pad for messages of length $\lambda$.
    
    Choose some $n \in \mathbb{Z}^{+}$. We will utilize $\Pi$ and $n$ to construct an new encryption scheme $\Pi^{*} = (Gen,Enc^{*},Dec^{*})$ with $\mathcal{K}^{*} = \mathcal{K}$ and $\lambda + n = |\mathcal{M}^{*}| = |\mathcal{C}^{*}|$. 
    
    $Gen$ in the same in both $\Pi$ and $\Pi^{*}$. 
    
    $Enc^{*}(m) : \mathcal{M} \to \mathcal{C} = Enc(m_{[0,\lambda-1]}) \,||\, m_{[\lambda,n-1]}$; applying $Enc$ on the first $\lambda$ bits of $m$ and leaving the last $n$ bits of $m$ unchanged. 
    
    $Dec^{*}(c) : \mathcal{C} \to \mathcal{M} = Dec(c_{[0,\lambda-1]}) \,||\, c_{[\lambda,n-1]}$; applying $Dec$ on the first $\lambda$ bits of $c$ and leaving the last $n$ bits of $c$ unchanged.
    
    We can see, trivially that $\Pi^{*}$ is not a perfectly secret encryption scheme as it leaks the trailing $n$ bits of the message, violating the requirement of perfect indistinguishablity. However, $\forall\, k \in \mathcal{K}^{*}$ and all adversaries $\mathcal{A}$, the probability that the adversary's guessed key $K$ is equal to the actual key $k$ is $2^{-\lambda}$ because $\mathcal{K}^{*} = \mathcal{K}$. 
    
    Therefore we have shown that for all $\lambda \in \mathbb{Z}^{+}$ there exists a countably infinite number of encryption schemes which are not perfectly secure, yet no adversary can guess the key with probability better than $2^{-\lambda}$, where $\lambda$ is the size of the key.


	\clearpage
	\section*{Question 2}
	\HomeworkText{
		State whether they each function is negligible or not and give a brief explanation.
	}

	\subsection*{Part A}
	\HomeworkText{
		$f(n) = n^{-100}$
	}

	$f$ is \emph{not} negligible. Consider the polynomial function $g(n) = n^{102}$
	
	\begin{equation}
	\begin{split}
	\pdv{}{n} \frac{f(n)}{g^{-1}(n)} & = \pdv{}{n} \frac{n^{-100}}{n^{-102}} \\
	  & = \pdv{}{n} n^2 \\
	  & = 2n \\
	  & \ge 1
	\end{split}
	\end{equation}
	
	We can see that the inverse of $g$ decreases faster than $f$ as the derivative of their ratio is greater than or equal to $1$.

	\subsection*{Part B}
	\HomeworkText{
		$f(n) = 2^{-100 \log n}$
	}

	$f$ is \emph{not} negligible. Consider the following equality:
	
	$$ f(n) = 2^{-100 \log n} = n^{-100}$$
	
	We can see above from \textbf{Part A} that $f$ is not negligible.
	
	\subsection*{Part C}
	\HomeworkText{
		$f(n) = n^{-\log n}$
	}

	$f$ is negligible. The function $f$ is presented in the assigned textbook, \emph{``Introduction to Modern Cryptography, 2nd Edition''}, on page 49 as an example of a negligible function. However, instead of taking the book's assertion at face value without a proof, we will sketch out our own.\\
	
	Choose a constant $c \in \mathbb{R}^{+}$. Let $N = 2^{c+1}$. Then for all $n \ge N$ the inequality $n^{-\log n} < n^{-c}$ will hold.\\
	
	\begin{equation}
	\begin{split}
	  	n^{-\log n} & < n^{-c} \\
	  	(2^{c+1})^{-\log(2^{c+1})} & < (2^{c+1})^{-c} \\
	    (2^{c+1})^{-c - 1} & < (2^{c+1})^{-c} \\
	\end{split}
	\end{equation}


	\subsection*{Part D}
	\HomeworkText{
		$f(n) = poly(n) \cdot negl(n)$
	}

	$f$ is negligible. The function $f$ is defined as neglible by \textbf{Proposition 3.6(2)} presented in the assigned textbook, \emph{``Introduction to Modern Cryptography, 2nd Edition''}, on page 49. Let's sketch a a proof again.\\
	
	Assume $f(n) = g(n) \cdot negl(n)$, where $g(n)$ is some positive polynomial function, is not negligible. Note that by this definition neither $g(n)$ nor $g^{-1}(n)$ can be negligible. We quickly arrive at the following contradiction:
	
	\begin{equation}
	\begin{split}
	f(n) & = g(n) \cdot negl(n) \\
	g^{-1} \cdot f(n) & = g^{-1} \cdot g(n) \cdot negl(n) \\
	g^{-1} \cdot f(n) & = negl(n) \\
	\end{split}
	\end{equation}
	
	\clearpage
	\section*{Question 3}
	\HomeworkText{
		Show that $A$ returns the correct bit $b$ with probability $\frac{1}{2} + \frac{\epsilon}{2}$ if and only if $A$ can distinguish the distributions $D_0$, $D_1$ with probability $\epsilon$.
	}
	

	\clearpage
	\section*{Question 4}
	\HomeworkText{
		Let $p$ be an odd prime.
	}
	\subsection*{Part A}
	\HomeworkText{
		Find an efficient algorithm to check if a given $a \in \mathbb{Z}^{*}_{p}$ is a quadratic residue.
	}


	\subsection*{Part B}
	\HomeworkText{
		Describe an efficient algorithm to check if a given $g \in \mathbb{Z}^{*}_{p}$ is a generator. Assume that the algorithm is also given a factorization of $p - 1$.
	}


\end{document}