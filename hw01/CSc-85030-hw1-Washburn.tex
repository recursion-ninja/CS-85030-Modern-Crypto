%This is my super simple Real Analysis Homework template

\documentclass{article}
\usepackage{geometry}
\geometry{
	a4paper,
%	total={170mm,257mm},
	left=20mm,
	right=20mm,
	top=20mm,
}
\usepackage[utf8]{inputenc}
\usepackage[english]{babel}
\usepackage{mathrsfs} % https://www.ctan.org/pkg/mathrsfs
\usepackage{amsmath}
\usepackage[]{amsthm} %lets us use \begin{proof}
\usepackage[]{amssymb} %gives us the character \varnothing
\usepackage{listings}
\usepackage{enumitem}
\usepackage{censor}
\usepackage{comment}
\usepackage{physics}

\lstnewenvironment{scala}{\lstset{language=Scala}}{}

%%% REDACT the problem statements that the grader doesn't like!
%\newcommand{\HomeworkText}[1]{\xblackout{``#1``}\\}
%\excludecomment{HomeworkScala}
\newcommand{\HomeworkText}[1]{\textbf{``#1''}\\}
\lstnewenvironment{HomeworkScala}{\lstset{language=Scala}}{}


\title{CSc - 850 Modern Cryptography\\ Homework 1}
\author{Alex Washburn}
\date\today

\begin{document}
	\newgeometry{left=4.5cm,right=4.5cm}

	\maketitle
	\vspace*{4cm}

	\restoregeometry
	
	\clearpage
	\section*{Question 1}
	\HomeworkText{
		Prove or disprove the following:
	}

	\subsection*{Part A}
	\HomeworkText{
		There is a perfectly secret encryption scheme that reveals 99\% of the key.
	}
	

	\subsection*{Part B}
	\HomeworkText{
		There is an encryption scheme that is not perfectly secure, yet no adversary can guess the key with probability better than $2^{\lambda}$, where $\lambda$ is the size of the key.
	}


	\clearpage
	\section*{Question 2}
	\HomeworkText{
		State whether they each function is negligible or not and give a brief explanation.
	}

	\subsection*{Part A}
	\HomeworkText{
		$f(n) = n^{-100}$
	}

	$f$ is \emph{not} negligible. Consider the polynomial function $g(n) = n^{102}$
	
	\begin{equation}
	\begin{split}
	\pdv{}{n} \frac{f(n)}{g^{-1}(n)} & = \pdv{}{n} \frac{n^{-100}}{n^{-102}} \\
	  & = \pdv{}{n} n^2 \\
	  & = 2n \\
	  & \ge 1
	\end{split}
	\end{equation}
	
	We can see that the inverse of $g$ decreases faster than $f$ as the derivative of their ratio is greater than or equal to $1$.

	\subsection*{Part B}
	\HomeworkText{
		$f(n) = 2^{-100 \log n}$
	}

	$f$ is \emph{not} negligible. Consider the following equality:
	
	$$ f(n) = 2^{-100 \log n} = n^{-100}$$
	
	We can see above from \textbf{Part A} that $f$ is not negligible.
	
	\subsection*{Part C}
	\HomeworkText{
		$f(n) = n^{-\log n}$
	}

	Assume that $f$ is negligible. Then the following must hold for all polynomials $g$.
	
	\begin{equation}
\begin{split}
	  0 = & \lim\limits_{n\rightarrow\infty} g(n)*f(n) \\
	    = & \lim\limits_{n\rightarrow\infty} \frac{g(n)}{n^{\log n}} \\
\end{split}
\end{equation}



	Assume there exists a polynomial function $g$ such that it's inverse $g^{-1}$ decreases faster than $f$. Then there exists some $c$ such that for all $n > c$, $\pdv{}{n} \frac{f(n)}{g^{-1}(n)} > 1$.\\

	We know that $f$ is a non-negative, decreasing function so the following must hold:

	\begin{equation}\label{eq:diff}
	  \forall n\quad 0 < f(n) - f(n^2)
	\end{equation}
	
	Consider $m > c$.
	
	We know from our assumption that both $\pdv{}{n} \frac{f(m)}{g^{-1}(m)} > 1$ and $\pdv{}{n} \frac{f(m^2)}{g^{-1}(m^2)} > 1$. Combined with \ref{eq:diff}, we know:

    \begin{equation}
	\frac{\pdv{}{n}f(m) - \pdv{}{n}f(m^2)}{\pdv{}{n}g^{-1}(m) - \pdv{}{n}g^{-1}(m^2)} > 1
	\end{equation}
	
	Given that $g$ being a polynomial function, we know that $\forall n,\; g^{-1}(n) - g^{-1}(n^2) = poly(n)$\\
	
	We have what we need, let's proceed!
	
%	$$ 1 < \pdv{}{n} \frac{f(n)}{g^{-1}(n)} < \pdv{}{n} \frac{f(n+k)}{g^{-1}(n+k)} $$\

$$ \frac{\pdv{}{n} \left( f(m) - f(m^2) \right)}{ \pdv{}{n} \left( g^{-1}(m) - g^{-1}(m^2) \right) } > 1$$

	\begin{equation}
	\begin{split}
	\frac{\pdv{}{n}f(m) - \pdv{}{n}f(m^2)}{\pdv{}{n}g^{-1}(m) - \pdv{}{n}g^{-1}(m^2)} & = \frac{\pdv{}{n} \left( f(m) - f(m^2) \right)}{ \pdv{}{n} \left( g^{-1}(m) - g^{-1}(m^2) \right) } \\
	& = \frac{\pdv{}{n} \left( m^{-\log m} - m^{-2\log(m^2)} \right)}{ \pdv{}{n} \left( poly(m) \right) } \\
	n^{-2\log(n^2)} - n^{-\log n}  & < -1 * poly(n)\\
	\end{split}
	\end{equation}

	\begin{equation}
	\begin{split}
	1 < \pdv{}{n} \frac{f(n)}{g^{-1}(n)} & < \pdv{}{n} \frac{f(n+k)}{g^{-1}(n+k)} \\
	1 < \frac{ n^{-1} * (-2 \log n - 1)}{n^{\log(n) - 1} g^{-1}(n)} & < \pdv{}{n} \frac{f(n+k)}{g^{-1}(n+k)} \\
	\end{split}
	\end{equation}

	\begin{equation}
	\begin{split}
	1 & \le	\pdv{}{n} \frac{f(n)}{g^{-1}(n)} \\
	  & =   -1 * \frac{n^{-\log n} * \pdv{}{n} g^{-1}(n)}{g^{-1}(n) * g^{-1}(n)} - \frac{2\log n * n^{-\log n - 1}}{g^{-1}(n)} \\
	  & = \frac{-1}{g^{-1}(n)} * \left[ \frac{\pdv{}{n} g^{-1}(n)}{n^{\log n} * g^{-1}(n)} + \frac{2\log n}{n^{\log n + 1}} \right] \\
	  & \le \pdv{}{n} \frac{f(n+k)}{g^{-1}(n+k)} \\
	  & = \frac{-1}{g^{-1}(n+k)} * \left[ \frac{\pdv{}{n} g^{-1}(n+k)}{(n+k)^{\log(n+k)} * g^{-1}(n+k)} + \frac{2\log(n+k)}{(n+k)^{\log(n+k) + 1}} \right]
	\end{split}
	\end{equation}
%		& =   \pdv{}{n} \frac{n^{-\log n}}{g^{-1}(n)} \\
%		& =   -1 * \frac{n^{-\log n} * \pdv{}{n} g^{-1}(n)}{g^{-1}(n) * g^{-1}(n)} - \frac{2\log n * n^{-\log n - 1}}{g^{-1}(n)} \\
%		& =  \frac{-1}{g^{-1}(n)} * \left[ \frac{\pdv{}{n} g^{-1}(n)}{n^{\log n} * g^{-1}(n)} + \frac{2\log n}{n^{\log n + 1}} \right] \\



We can see that the inverse of $g$ decreases faster than $f$ as the derivative of their ratio is greater than or equal to $1$.


	\subsection*{Part D}
	\HomeworkText{
		$f(n) = poly(n) \times negl(n)$
	}


	\clearpage
	\section*{Question 3}
	\HomeworkText{
		Show that $A$ returns the correct bit $b$ with probability $\frac{1}{2} + \frac{\epsilon}{2}$ if and only if $A$ can distinguish the distributions $D_0$, $D_1$ with probability $\epsilon$.
	}
	

	\clearpage
	\section*{Question 4}
	\HomeworkText{
		Let $p$ be an odd prime.
	}
	\subsection*{Part A}
	\HomeworkText{
		Find an efficient algorithm to check if a given $a \in \mathbb{Z}^{*}_{p}$ is a quadratic residue.
	}


	\subsection*{Part B}
	\HomeworkText{
		Describe an efficient algorithm to check if a given $g \in \mathbb{Z}^{*}_{p}$ is a generator. Assume that the algorithm is also given a factorization of $p - 1$.
	}


\end{document}